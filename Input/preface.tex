\clearpage
				
\setcounter{page}{1}

\chapter*{Preface}
\addcontentsline{toc}{chapter}{Preface}	

The work presented in the thesis has been carried out at the Department of Engineering Cybernetics at the Norwegian University of Science and Technology. This thesis concerns the application of model predictive control to embedded systems, in particular a helicopter simulator at the university, mostly used in the lab work of students taking the subject TTK4135 Optimization and Control.

The initial plan for my thesis was a different, more theoretical idea researching a specific variation of MPC and simulating it. However, after a month of researching this and struggling, my advisor and I came to the conclusion that it was a more difficult task than originally assumed. So six weeks after the beginning of the masters, the scope and goal of the thesis shifted drastically. 

The new idea was to use the work from the previous semester, involving simulating a system with tube MPC, and add a practical twist, which was applying this method to the helicopter. However, after a few weeks, I quickly realized due to the number of states of the system and the computational complexity of the implementation of the tube MPC, this would not be possible. 

So the result, as seen in this thesis, is controlling the helicopter using both nominal and stable MPC.

The motivation for choosing this project was mostly wanting to apply my work from the previous semester to a practical example. In addition, having taken the course TTK4135, finding a way to include MPC, which is a large part of the curriculum, into the lab work required in this class could have a very positive outcome of the learning experience of future students.

All work is carried out at NTNU, in the helicopter lab.

The reader is assumed  to have some understanding of embedded systems, optimization, Simulink, MATLAB and MPC.

The work discussed in this report is conducted within [...]. It is also based off the work conducted by the author in [...].

The basis for the MATLAB simulation was developed in \cite{prosjekt_oppgave} last semester. It has been changed to include the \acrshort{qp}-solver OSQP.

The work has been conducted with MATLAB R2015b and Simulink.

First and foremost, I would like to thank my advisor, Professor Lars Imsland, for being as excited about the work done as I was. I would also like to thank Joakim Rostrup Andersen for [...]. 

Most importantly, I would like to thank my family for having supported me throughout these five years. Though not every subject and every issue has been understood, I have always gotten the motivation and support I've needed from you all.


