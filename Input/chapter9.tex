\chapter{Conclusion}

The research question for this thesis was in short \textit{What are the effects of controlling a 3 DOF helicopter using linear \acrlong{mpc}?}

Conclusively, using \acrshort{mpc} to control a propeller under-actuated system such as the Quanser's 3 DOF Helicopter is both a useful area of research, and it also showed a few things. 

The performance difference between linear \acrlong{mpc} with and without the terminal cost and constraint was very apparent in this system. In addition, the attempt to remove the offset by modelling the constant disturbance was relatively successful. 

The work done on this piece of equipment is extensive, and the work presented here will only add to that. 

Going forward, adding trajectory tracking would be a natural next step. Nonlinear \acrlong{mpc} could also have some interesting application, both academically and for application to larger, more complex systems. In addition, it is apparent that robust control of the system is needed, as there a many disturbances that are not modelled here. Robust linear control would be less complex than non-linear control, where there would still be disturbances that are not modelled. 
